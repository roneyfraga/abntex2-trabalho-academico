% ----------------------------------------------------------
% VERSÃO ORIGINAL
% ----------------------------------------------------------
% The Current Maintainer of this work is the abnTeX2 team, led
% by Lauro César Araujo. Further information are available on
% http://abntex2.googlecode.com/

\documentclass[
	% -- opções da classe memoir --
	12pt,				% tamanho da fonte
	openright,			% capítulos começam em pág ímpar (insere página vazia caso preciso)
	oneside,			% para impressão em verso e anverso coloque twoside
	a4paper,			% tamanho do papel.
	% -- opções da classe abntex2 --
	%chapter=TITLE,		% títulos de capítulos convertidos em letras maiúsculas
	%section=TITLE,		% títulos de seções convertidos em letras maiúsculas
	%subsection=TITLE,	% títulos de subseções convertidos em letras maiúsculas
	%subsubsection=TITLE,% títulos de subsubseções convertidos em letras maiúsculas
	% -- opções do pacote babel --
	french,				% idioma adicional para hifenização
	spanish,			% idioma adicional para hifenização
	english,			% idioma adicional para hifenização
	brazil				% o último idioma é o principal do documento
	]{abntex2}


% ----------------------------------------------------------
% PACOTES
% ----------------------------------------------------------
% ----------------------------------------------------------
% PACOTES BÁSICOS
% ----------------------------------------------------------
\usepackage{lmodern}            % Usa a fonte Latin Modern          
\usepackage[T1]{fontenc}        % Selecao de codigos de fonte.
\usepackage[utf8]{inputenc}     % Codificacao do documento (conversão automática dos acentos)
\usepackage{lastpage}           % Usado pela Ficha catalográfica
\usepackage{indentfirst}        % Indenta o primeiro parágrafo de cada seção.
\usepackage{color}              % Controle das cores
\usepackage{graphicx}           % Inclusão de gráficos
\usepackage{microtype}          % para melhorias de justificação
\usepackage{listings}           % Inserir código de linguagem de programação
\usepackage{nomencl}            % Necessário para o commando makeindex
        
% Pacotes de citações
\usepackage[brazilian,hyperpageref]{backref}     % Paginas com as citações na bibl
\usepackage[alf]{abntex2cite}   % Citações padrão ABNT

% CONFIGURAÇÕES DE PACOTES
% Configurações do pacote backref
% Usado sem a opção hyperpageref de backref
\renewcommand{\backrefpagesname}{Citado na(s) página(s):~}
% Texto padrão antes do número das páginas
\renewcommand{\backref}{}
% Define os textos da citação
\renewcommand*{\backrefalt}[4]{
    \ifcase #1 %
        Nenhuma citação no texto.%
    \or
        Citado na página #2.%
    \else
        Citado #1 vezes nas páginas #2.%
    \fi}%

% Pacotes adicionais, usados apenas no âmbito do Modelo Canônico do abnteX2
\usepackage{lipsum}             % para geração de dummy text

% TODO inserir seus pacotes aqui



% ----------------------------------------------------------
% CAPA E FOLHA DE ROSTO
% ----------------------------------------------------------
% ----------------------------------------------------------
% CAPA E FOLHA DE ROSTO
% ----------------------------------------------------------
\titulo{Modelo Canônico de\\ Trabalho Acadêmico com \abnTeX}
\autor{Equipe \abnTeX}
\local{Brasil}
\data{2014, v-1.9.2}
\orientador{Lauro César Araujo}
\coorientador{Equipe \abnTeX}
\instituicao{%
  Universidade do Brasil -- UBr
  \par
  Faculdade de Arquitetura da Informação
  \par
  Programa de Pós-Graduação}
\tipotrabalho{Tese (Doutorado)}
% O preambulo deve conter o tipo do trabalho, o objetivo, 
% o nome da instituição e a área de concentração 
\preambulo{Modelo canônico de trabalho monográfico acadêmico em conformidade com
as normas ABNT apresentado à comunidade de usuários \LaTeX.}



% ----------------------------------------------------------
% CONFIGURAÇÕES
% ----------------------------------------------------------

% Configurações de aparência do PDF final

% alterando o aspecto da cor azul
\definecolor{blue}{RGB}{41,5,195}

% informações do PDF
\makeatletter
\hypersetup{
        %pagebackref=true,
        pdftitle={\@title},
        pdfauthor={\@author},
        pdfsubject={\imprimirpreambulo},
        pdfcreator={LaTeX with abnTeX2},
        pdfkeywords={abnt}{latex}{abntex}{abntex2}{trabalho acadêmico},
        colorlinks=true,            % false: boxed links; true: colored links
        linkcolor=blue,             % color of internal links
        citecolor=blue,             % color of links to bibliography
        filecolor=magenta,              % color of file links
        urlcolor=blue,
        bookmarksdepth=4
}
\makeatother

% Espaçamentos entre linhas e parágrafos
% O tamanho do parágrafo é dado por:
\setlength{\parindent}{1.3cm}

% Controle do espaçamento entre um parágrafo e outro:
\setlength{\parskip}{0.2cm}  % tente também \onelineskip

% compila o indice
\makeindex
\makenomenclature

% ----------------------------------------------------------
% INÍCIO DOCUMENTO
% ----------------------------------------------------------
\begin{document}

% Retira espaço extra obsoleto entre as frases.
\frenchspacing

% ----------------------------------------------------------
% ELEMENTOS PRÉ-TEXTUAIS
% ----------------------------------------------------------
% \pretextual

% Capa
\imprimircapa

% Folha de rosto
% (o * indica que haverá a ficha bibliográfica)
\imprimirfolhaderosto*

\input{elementos-pretextuais/ficha-catalografica}
\input{elementos-pretextuais/errata}
\input{elementos-pretextuais/folha-aprovacao}
\input{elementos-pretextuais/dedicatoria}
\input{elementos-pretextuais/agradecimentos}
\input{elementos-pretextuais/epigrafe}
\input{elementos-pretextuais/resumos}
\input{elementos-pretextuais/epigrafe}


% ----------------------------------------------------------
% inserir lista de ilustrações
% ----------------------------------------------------------
\pdfbookmark[0]{\listfigurename}{lof}
\listoffigures*
\cleardoublepage

% ----------------------------------------------------------
% inserir lista de tabelas
% ----------------------------------------------------------
\pdfbookmark[0]{\listtablename}{lot}
\listoftables*
\cleardoublepage

% ----------------------------------------------------------
% inserir lista siglas e abreviaturas
% ----------------------------------------------------------
\input{elementos-pretextuais/siglas}

% ----------------------------------------------------------
% inserir lista símbolos
% ----------------------------------------------------------
% INSERIR LISTA DE SÍMBOLOS
% ----------------------------------------------------------
\begin{simbolos}
  \item[$ \Gamma $] Letra grega Gama
  \item[$ \Lambda $] Lambda
  \item[$ \zeta $] Letra grega minúscula zeta
  \item[$ \in $] Pertence
\end{simbolos}


% ----------------------------------------------------------

% ----------------------------------------------------------
% inserir o sumario
% ----------------------------------------------------------
\pdfbookmark[0]{\contentsname}{toc}
\tableofcontents*
\cleardoublepage

% ----------------------------------------------------------
% ELEMENTOS TEXTUAIS
% ----------------------------------------------------------
\textual

% ----------------------------------------------------------
% Introdução (exemplo de capítulo sem numeração, mas presente no Sumário)
% ----------------------------------------------------------
% TODO inserir seu capítulo 1 aqui
\chapter*[Introdução]{Introdução}
\addcontentsline{toc}{chapter}{Introdução}
% ----------------------------------------------------------
\include{elementos-textuais/introducao}

% ----------------------------------------------------------
% PARTE
% ----------------------------------------------------------
\part{Preparação da pesquisa}

% ----------------------------------------------------------
% Capitulo com exemplos de comandos inseridos de arquivo externo
% ----------------------------------------------------------
\include{elementos-textuais/capitulo-1}

% ----------------------------------------------------------
% PARTE
% ----------------------------------------------------------
\part{Dicas de úteis}

% ----------------------------------------------------------
% Capitulo com exemplos de comandos inseridos de arquivo externo
% ----------------------------------------------------------
\chapter{Dicas úteis}\label{cap_dicas}

O presente capítulo foi escrito foi Roney (https://github.com/roneyfraga), que não faz parte da equipe da Equipe \abnTeX . Apenas algumas dicas serão acrescentadas.

\section{Aprender \LaTeX }
Se você não sabe por onde começar a estudar para aprender \LaTeX, segue lista de materiais que eu utilizei e que sempre recorro quando preciso:

\begin{itemize}
    \item \href{http://www.mat.ufmg.br/~regi/topicos/intlat.pdf}{Introdução ao \LaTeX} do Professor Reginaldo J. Santos.
    \item \href{http://zelmanov.ptep-online.com/ctan/lshort_port.pdf}{Uma introdução não tão pequena de \LaTeX} por Tobias Oetiker Hubert Partl, Irene Hyna e Elisabeth Schlegl. Tradução de Alberto Simões.
    \item \href{http://www.uel.br/projetos/matessencial/superior/pdfs/latexmat.pdf}{\LaTeX\ para matemática com o TeXnicCenter} de Ulysses Sodré.
    \item \href{http://en.wikibooks.org/wiki/LaTeX}{\LaTeX\ Wikibooks.}
\end{itemize}

\section{\LaTeX + R}

\begin{itemize}
    \item knitr
    \item tikz
    \item tables
    \item xtable
\end{itemize}

\section{\LaTeX + Referências}

\begin{itemize}
    \item zotero
    \item mendley
    \item google
    \item sites de revistas
\end{itemize}

\section{\LaTeX + diversos}


% ----------------------------------------------------------

% Finaliza a parte no bookmark do PDF
% para que se inicie o bookmark na raiz
% e adiciona espaço de parte no Sumário
% ----------------------------------------------------------
\phantompart

% ----------------------------------------------------------
% Conclusão (outro exemplo de capítulo sem numeração e presente no sumário)
% ----------------------------------------------------------
\chapter*[Conclusão]{Conclusão}
\addcontentsline{toc}{chapter}{Conclusão}
% ----------------------------------------------------------
\input{elementos-textuais/conclusao}

% ----------------------------------------------------------
% ELEMENTOS PÓS-TEXTUAIS
% ----------------------------------------------------------
\postextual
% ----------------------------------------------------------

% ----------------------------------------------------------
% Referências bibliográficas
% ----------------------------------------------------------
\bibliography{elementos-postextuais/referencias}

% ----------------------------------------------------------
% Glossário
% ----------------------------------------------------------
%
% Consulte o manual da classe abntex2 para orientações sobre o glossário.
%
%\glossary

% ----------------------------------------------------------
% Apêndices
% ----------------------------------------------------------
\input{elementos-postextuais/apendices}


% ----------------------------------------------------------
% Anexos
% ----------------------------------------------------------
\input{elementos-postextuais/anexos}

%---------------------------------------------------------------------
% INDICE REMISSIVO
%---------------------------------------------------------------------
\phantompart
\printindex
%---------------------------------------------------------------------

\end{document}
